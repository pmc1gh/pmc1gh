\newif\ifshowsolutions
\showsolutionstrue
\documentclass{article}
\usepackage{listings}
\usepackage{amsmath}
\usepackage{subfig}
\usepackage{amsthm}
\usepackage{amsmath}
\usepackage{amssymb}
\usepackage{graphicx}
\usepackage{mdwlist}
\usepackage{geometry}
\usepackage{titlesec}
\usepackage{palatino}
\usepackage{mathrsfs}
\usepackage{fancyhdr}
\usepackage{paralist}
\usepackage{todonotes}
\usepackage{tikz}
\usepackage{float} % Place figures where you ACTUALLY want it
\usepackage{comment} % A hack to toggle sections
\usepackage{ifthen}
\usepackage{mdframed}
\usepackage{verbatim}
\usepackage{listings}
\usepackage{bbm}
\usepackage{upquote} % Prevents backticks replacing single-quotes in verbatim
\usepackage[strings]{underscore}
\usepackage[colorlinks=true]{hyperref}
\usetikzlibrary{positioning,shapes,backgrounds}

\geometry{margin=1in}
\geometry{headheight=2in}
\geometry{top=2in}

\setlength{\marginparwidth}{2.15cm}
\setlength{\parindent}{0em}
\setlength{\parskip}{0.6\baselineskip}

\rhead{}
\lhead{}

% Spacing settings.
\titlespacing\section{0pt}{12pt plus 2pt minus 2pt}{0pt plus 2pt minus 2pt}
\titlespacing\subsection{0pt}{12pt plus 4pt minus 2pt}{0pt plus 2pt minus 2pt}
\titlespacing\subsubsection{0pt}{12pt plus 4pt minus 2pt}{0pt plus 2pt minus 2pt}
\renewcommand{\baselinestretch}{1.15}

% Shortcuts for commonly used operators.
\newcommand{\E}{\mathbb{E}}
\newcommand{\Var}{\operatorname{Var}}
\newcommand{\Cov}{\operatorname{Cov}}
\newcommand{\Bias}{\operatorname{Bias}}
\DeclareMathOperator{\argmin}{arg\,min}
\DeclareMathOperator{\argmax}{arg\,max}

% Do not number subsections and below.
\setcounter{secnumdepth}{1}

% Custom format subsection.
\titleformat*{\subsection}{\large\bfseries}

% Set up the problem environment.
\newcounter{problem}[section]
\newenvironment{problem}[1][]
  {\begingroup
    \setlength{\parskip}{0em}
    \refstepcounter{problem}\par\addvspace{1em}\textbf{Problem~\Alph{problem}\!
    \ifthenelse{\equal{#1}{}}{}{ [#1 points]}:}
  \endgroup}

% Set up the subproblem environment.
\newcounter{subproblem}[problem]
\newenvironment{subproblem}[1][]
  {\begingroup
    \setlength{\parskip}{0em}
    \refstepcounter{subproblem}\par\medskip\textbf{\roman{subproblem}.\!
    \ifthenelse{\equal{#1}{}}{}{ [#1 points]:}}
  \endgroup}

% Set up the teachers and materials commands.
\newcommand\teachers[1]
  {\begingroup
    \setlength{\parskip}{0em}
    \vspace{0.3em} \textit{\hspace*{2em} TAs responsible: #1} \par
  \endgroup}
\newcommand\materials[1]
  {\begingroup
    \setlength{\parskip}{0em}
    \textit{\hspace*{2em} Relevant materials: #1} \par \vspace{1em}
  \endgroup}

% Set up the hint environment.
\newenvironment{hint}[1][]
  {\begin{em}\textbf{Hint: }}
  {\end{em}}

% Set up the solution environment.
\ifshowsolutions
  \newenvironment{solution}[1][]
    {\par\medskip \begin{mdframed}\textbf{Solution~\Alph{problem}#1:} \begin{em}}
    {\end{em}\medskip\end{mdframed}\medskip}
  \newenvironment{subsolution}[1][]
    {\par\medskip \begin{mdframed}\textbf{Solution~\Alph{problem}#1.\roman{subproblem}:} \begin{em}}
    {\end{em}\medskip\end{mdframed}\medskip}
\else
  \excludecomment{solution}
  \excludecomment{subsolution}
\fi



%%%%%%%%%%%%%%%%%%%%%%%%%%%%%%
% HEADER
%%%%%%%%%%%%%%%%%%%%%%%%%%%%%%

\chead{
  {\vbox{
      \vspace{2mm}
      \large
      Machine Learning \& Data Mining \hfill
      Caltech CS/CNS/EE 155 \hfill \\[1pt]
      Set 5\hfill
      February 2019\\
    }
  }
}

\begin{document}
\pagestyle{fancy}



\section{SVD and PCA [35 Points]}

\problem[3] 

\begin{solution}
\end{solution}

\newpage
\problem[4] 

\begin{solution}
 
\end{solution}

\newpage
\problem[5] 

\begin{solution}

\end{solution}

\newpage
\problem[3] 

\begin{solution}

\end{solution}


\newpage
\problem[3] .

\begin{solution}

\end{solution}

\newpage
\problem[3] 

\begin{solution}

\end{solution} 

\newpage
\problem[4] 

\begin{solution}

\end{solution}


\newpage
\problem[4] 

\begin{solution}

\end{solution}

\newpage
\problem[4] 
\begin{solution}
\end{solution}

\newpage
\problem[2] 
\begin{solution}

\end{solution}


\newpage
\section{Matrix Factorization [30 Points]}

\problem[5]

\begin{solution}

\end{solution}

\newpage
\problem[5]

\begin{solution}

\end{solution}

\newpage
\problem[10]

\begin{solution}
See 2D.py and prob2utils.py for the solution code.
\end{solution}

\newpage
\problem[5]

\begin{solution}

%
%\begin{figure}[H]
%\begin{center}
%\includegraphics[width=0.8\textwidth]{plots/2d.png}
%\caption{Unregularized factorization}
%\label{fig:UnregFact}
%\end{center}
%\end{figure}


\end{solution}

\newpage
\problem[5]

\begin{solution}


%\begin{figure}[H]
%\begin{center}
%\includegraphics[width=0.45\textwidth]{plots/2e_ein.png}
%\caption{$E_{in}$ vs $k$ for different $\lambda$}
%\label{fig:RegFact1}
%\end{center}
%\end{figure}

%\begin{figure}[H]
%\begin{center}
%\includegraphics[width=0.45\textwidth]{plots/2e_eout.png}
%\caption{$E_{out}$ vs $k$ for different $\lambda$}
%\label{fig:RegFact2}
%\end{center}
%\end{figure}

 
\end{solution}






\newpage
\section{Word2Vec Principles [35 Points]}

\problem[5]


\begin{solution}

\end{solution}


\newpage
\problem[10]

\begin{solution}

\end{solution}


\newpage
\problem[3]


\begin{solution}

\end{solution}




\newpage
\problem[10]



\begin{solution}
See solution code in P3C.py
\end{solution}


\newpage
\problem[2]


\begin{solution}

\end{solution}

\newpage
\problem[2]

\begin{solution}

\end{solution}

\newpage
\problem[1]

\begin{solution}

%\begin{verbatim}
%
%\end{verbatim}

\end{solution}
\newpage

\problem[2]

\begin{solution}


\end{solution}

\end{document}

